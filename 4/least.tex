\section{Principle of Least Action; Noether's Theorem}
Consider a particle in $\mathbb R^3$ with kinetic energy and potential energy denotes by $T,V$ respectively.
Define the Lagrangian to be $L(\underline{x},\underline{\dot{x}},t)=T-V$ and the action
$$S[\underline{x}]=\int_{t_1}^{t_2}L\,\mathrm dt$$
Hamilton's Principle (or Principle of Least Action) postulates that the motion of the particle is a stationary point of the action functional $S$, i.e. $\underline{x}$ satisfies the Euler-Lagrange equations with the integrand being $L$.
\begin{example}
    Take the kinetic energy to be $T=m|\underline{\dot{x}}|^2/2$ and $V=V(\underline{x})$, then the Euler-Lagrange equation gives
    $$\frac{\mathrm d}{\mathrm dt}\frac{\partial f}{\partial \dot{x}_i}-\frac{\partial L}{\partial x_i}=0$$
    for any $i$.
    This simplifies to
    $$m\ddot{x}_i=-\frac{\partial V}{\partial x_i}$$
    which is just Newton's Second Law.
\end{example}
\begin{example}[Central Force in Two Dimensions]
    We have
    $$L=\frac{1}{2}m(\dot{r}^2+r^2\dot{\theta}^2)-V(r)$$
    So the Euler-Lagrange equations give
    $$\frac{\mathrm d}{\mathrm dt}\frac{\partial L}{\partial \dot{r}}-\frac{\partial L}{\partial r}=0=\frac{\mathrm d}{\mathrm dt}\frac{\partial L}{\partial \dot{\theta}}-\frac{\partial L}{\partial \theta}=\frac{\mathrm d}{\mathrm dt}\frac{\partial L}{\partial \dot{\theta}}$$
    The first observation is that the rightmost expression vanishes, hence
    $$mr^2\dot{\theta}=\frac{\partial L}{\partial\dot{\theta}}=\text{const.}$$
    which is just the conservation of angular momentum.
    Now $\partial L/\partial t=0$, so by our previous discussion of the first integrals,
    $$\dot{r}\frac{\partial L}{\partial \dot{r}}+\dot{\theta}\frac{\partial L}{\partial\dot{\theta}}-L=\text{const.}$$
    So by substitution and a bit of simplification, we are left with
    $$\frac{1}{2}m\dot{r}^2+\frac{1}{2}mr^2\dot{\theta}^2+V(r)=\text{const.}$$
    Note that the left hand side is precisely $L+2V=E+V$, so this can be viewed as the conservation of energy.
\end{example}
\begin{example}[Configuration Space and Generalised Coordinates]
    Consider $N$ particles in $\mathbb R^3$.
    We can put the information of the coordinates of the particles as a $3N$-tuple, which allows us to take the $N$ particles as a single particle in $\mathbb R^{3N}$.
    The coordinates of all $N$ particles ordered in the obvious way is called a generalised coordinate, and the particle system described in $\mathbb R^{3N}$ makes it a configuration space.
    So the motion of the particles can be characterised by a path (i.e. the motion of one single particle) in $\mathbb R^{3N}$, i.e. $t\mapsto (q_i,\dot{q}_i,t)$ where $q_i$ is the $i^{th}$ generalised coordinate, $i=1,\ldots,3N$.
    The Lagrangian considered this way is then $L=L(q_i,\dot{q}_i,t)$ which we can solve the corresponding Euler-Lagrange equation for.
    There is a perspective of classical dynamics that takes this point of view to solve for the motions of the system of particles.
\end{example}
Consider the functional
$$F[\underline{y}]=\int_\alpha^\beta f(y_i,y_i^\prime,x)\,\mathrm dx,i=1,\ldots,n$$
Consider a $1$-parameter family of transformations
$$y_i(x)\mapsto Y_i(s,x),s\in\mathbb R$$
We call this a continuous symmetry (or simply symmetry) of a Lagrangian if
$$\frac{\mathrm d}{\mathrm ds}f(Y_i(s,x),Y^\prime(s,x),x)=0$$
With these set-ups, we have
\begin{theorem}[Noether's Theorem, simple version]
    Given a continuous symmetry $Y_i(s,x)$ of $f$ with $Y_i(x,0)=y_i(x)$ for all $i$, the quantity
    $$\sum_i\frac{\partial f}{\partial y_i}\left.\frac{\partial Y_i}{\partial s}\right|_{s=0}$$
    is a first integral of the Euler-Lagrange equations (i.e. is constant).
\end{theorem}
\begin{proof}
    Below by $f$ we mean $f(Y_i(s,x),Y^\prime(s,x),x)$.
    Summation convention is used.
    \begin{align*}
        0&=\left.\frac{\mathrm df}{\mathrm ds}\right|_{s=0}\\
        &=\frac{\partial f}{\partial y_i}\left.\frac{\partial Y_i}{\partial s}\right|_{s=0}+\frac{\partial f}{\partial y_i^\prime}\left.\frac{\partial Y_i^\prime}{\partial s}\right|_{s=0}\\
        &=\left.\frac{\partial Y_i}{\partial s}\right|_{s=0}\frac{\mathrm d}{\mathrm dx}\frac{\partial f}{\partial y_i}+\frac{\partial f}{\partial y_i^\prime}\left.\frac{\mathrm d}{\mathrm dx}\frac{\partial Y_i}{\partial s}\right|_{s=0}\\
        &=\frac{\mathrm d}{\mathrm dx}\left( \frac{\partial f}{\partial y_i}\left.\frac{\partial Y_i}{\partial s}\right|_{s=0} \right)
    \end{align*}
    The theorem follows.
\end{proof}
\begin{example}
    Consider
    $$f=\frac{1}{2}(y^\prime)^2+\frac{1}{2}(z^\prime)^2-V(y-z)$$
    which is the Lagrangian of a potential that depends only on difference of the components.
    Consider $Y=y+s, Z=z+s$, then $Y^\prime=y^\prime$ and $Z^\prime=z^\prime$ and $V(Y-Z)=V(y-z)$, so indeed it is a continuous symmetry.
    Noether's Theorem then implies that
    $$\left.\left( \frac{\partial f}{\partial y^\prime}\frac{\partial Y}{\partial s}+\frac{\partial f}{\partial z^\prime}\frac{\partial Z}{\partial s} \right)\right|_{s=0}=y^\prime+z^\prime$$
    is constant.
    This is just saying the conservation of momentum in the $y+z$ direction.
\end{example}
\begin{example}
    Back to our previous discussion on central force, then the transformation $\Theta=\theta+s,R=r$ is certainly a continous symmetry.
    Therefore we have the conserved quantity
    $$\left( \frac{\partial L}{\partial\dot{\theta}}\frac{\partial\Theta}{\partial s}+\frac{\partial L}{\partial\dot{r}}\frac{\partial R}{\partial s} \right)_{s=0}=mr^2\dot{\theta}$$
    which is just the magnitude of angular momentum.
    In general, isotopy of space gives rise to rotational invariants of the Lagrangian.
\end{example}
