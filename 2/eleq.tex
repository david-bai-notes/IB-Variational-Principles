\section{The Euler-Lagrange Equation}
\subsection{Derivation of the Equation}
We now move on to the most important theorem of the course, which gives a necessary condition to extremise a functional in the form
$$F[y]=\int_\alpha^\beta f(x,y,y^\prime)\,\mathrm dx$$
where $f$ is given.
In contexts of geometry and physics, $y$ is most likely to carry the meaning of the trajectory of some point.\\
We first assume that a extremum $y$ exists, then when we apply a small perturbation $y\mapsto y+\epsilon\eta$ with $\eta(\alpha)=\eta(\beta)=0$ to keep the endpoint fixed.
Now we want to compute $F[y+\epsilon\eta]$, but first of all we will need a lemma.
\begin{lemma}\label{fund_lemma}
    If $g:[\alpha,\beta]\to\mathbb R$ is continuous on $[\alpha,\beta]$ and
    $$\int_\alpha^\beta g(x)\eta(x)\,\mathrm dx=0$$
    for all $\eta\in C([\alpha,\beta])$ with $\eta(\alpha)=\eta(\beta)=0$, then $\forall x\in [\alpha,\beta],g(x)=0$.
\end{lemma}
\begin{proof}
    Assume for sake of contradiction that there exists some $\bar{x}$ on $(\alpha,\beta)$ such that $g(\bar{x})\neq 0$.
    WLOG $g(\bar{x})>0$, then by continuity there is an interval $[x_1,x_2]\subset[\alpha,\beta]$ such that $\exists c>0,\forall x\in [x_1,x_2],g(x)>c$.
    Set
    $$\eta(x)=\begin{cases}
        (x-x_1)(x_2-x)\text{, if $x\in [x_1,x_2]$}\\
        0\text{, otherwise}
    \end{cases}$$
    Then
    \begin{align*}
        \int_\alpha^\beta g(x)\eta(x)\,\mathrm dx
        &=\int_{x_1}^{x_2}g(x)(x-x_1)(x_2-x)\,\mathrm dx\\
        &\ge\int_{x_1}^{x_2}c(x-x_1)(x_2-x)\,\mathrm dx\\
        &>0
    \end{align*}
    Contradiction.
    So $g$ is zero on $(\alpha,\beta)$, and it is also zero at $\alpha,\beta$ by continuity, hence $g$ is zero on $[\alpha,\beta]$.
\end{proof}
The $\eta$ we have used above is called a bump function, which is $C^2$ as one can verify.
One can also make a $C^k$ bump function by considering
\footnote{It is also easy to construct a $C^{\infty}$ bump function.}
$$\eta(x)=\begin{cases}
    ((x-x_1)(x_2-x))^{k+1}\text{, if $x\in [x_1,x_2]$}\\
    0\text{, otherwise}
\end{cases}$$
Now back at $F[y+\epsilon\eta]$, then
\begin{align*}
    F[y+\epsilon\eta]&=\int_\alpha^\beta f(x,y+\epsilon\eta,y^\prime+\epsilon\eta^\prime)\,\mathrm dz\\
    &=F[y]+\epsilon\int_\alpha^\beta\left( \frac{\partial f}{\partial y}\eta+\frac{\partial f}{\partial y^\prime}\eta^\prime \right)\,\mathrm dx+O(\epsilon^2)
\end{align*}
We will analysis the $O(\epsilon^2)$ remainder later.
For now, we just observe that for $y$ to be an extremum, the first-order term shall vanish, so we want something like $\partial F[y+\epsilon\eta]/\partial\epsilon=0$.\\
Integrate the vanishing first-order coefficient by part,
\begin{align*}
    0&=\left.\frac{\partial f}{\partial y^\prime}\eta\right|_\alpha^\beta+\int_\alpha^\beta\left( \frac{\partial f}{\partial y}\eta-\frac{\mathrm d}{\mathrm dx}\frac{\partial f}{\partial y^\prime}\eta \right)\,\mathrm dx\\
    &=\int_\alpha^\beta\left( \frac{\partial f}{\partial y}-\frac{\mathrm d}{\mathrm dx}\frac{\partial f}{\partial y^\prime} \right)\eta\,\mathrm dx
\end{align*}
By the preceding lemma, we must have
$$\frac{\partial f}{\partial y}-\frac{\mathrm d}{\mathrm dx}\frac{\partial f}{\partial y^\prime}=0$$
This is known as the Euler-Lagrange equation, which is the necessary condition for an extremum.
This equation is first developed in 1745 in a letter from Lagrange to Euler.
\begin{remark}
    1. The Euler-Lagrange equation is a second-order ODE with initial conditions $y(\alpha)=y_1,y(\beta)=y_2$.\\
    2. Sometimes the LHS is denoted $\delta F[y]/\delta y(x)$ and is called the functional derivative.
    Some author also write $\epsilon\eta=\delta y$, allowing one to write $F[y+\delta y]=F[y]+\delta F[y]$ where
    $$\delta F[y]=\int_\alpha^\beta\frac{\delta F[y]}{\delta y(x)}\delta y(x)\,\mathrm dx$$
    3. Other kinds of boundary conditions (e.g. those on $y^\prime$) are possible.\\
    4. Be careful with derivatives as the notation can be a bit confusing.
    The $x,y,y^\prime$ are independent variables when we are talking about partial derivatives of $f(x,y,y^\prime)$.\\
    5. For any $h(x,y,y^\prime)$, a somewhat useful formula is
    $$\frac{\mathrm dh(x,y(x),y^\prime(x))}{\,\mathrm dx}=\frac{\partial h}{\partial x}+\frac{\partial h}{\partial y}y^\prime+\frac{\partial h}{\partial y^\prime}y^{\prime\prime}$$
    For example, for $f(x,y,y^\prime)=x((y^\prime)^2-y^2)$, we have $\mathrm df/\mathrm dx=(y^\prime)^2-y^2-2xyy^\prime+2y^{\prime\prime}y^\prime x$.
\end{remark}
\subsection{First Integrals of the Euler-Lagrange Equation}
Now we move on to solve the Euler-Lagrange equations, which is a second order ODE.
In some special cases, this is a quite easy thing to do.
In particular, if $f$ does not explicitly depend on some of its variables, then we can simplify the equation to something that is easier to solve.
These simplifications are often in the form of something being constant.
Expressions like these are called the first integrals of the equations.
Not only are they tools we can use to solve the equation, we can also view them as a conserved quantity of something that is described by a variational problem.
We will discuss the former in this section.
The latter will be mentioned later, when we discuss Noether's Theorem.\\
Assume that $f$ does not explicitly depend on $y$, then $\partial f/\partial y=0$, so the Euler-Lagrange equation can be rewritten as
$$\frac{\mathrm d}{\mathrm dx}\frac{\partial f}{\partial y^\prime}=0\implies \frac{\partial f}{\partial y^\prime}=\text{const.}$$
which is a first order ODE.
\begin{example}[Geodesics on the Plane]
    Consider the geodesics on the Euclidean plane.
    We know that we want to extremise the functional
    $$F[y]=\int_\alpha^\beta\sqrt{1+(y^\prime)^2}\,\mathrm dx$$
    In this case, the apparent $f$ does not depend on $y$, hence we can obtain the solution by just solving the first integral
    $$\frac{y^\prime}{\sqrt{1+(y^\prime)^2}}=\text{const.}$$
    One can solve this to get $y=mx+c$ for some constants $m,c$, which is our familiar formulation of a straight line, the geodesics of the plane.
\end{example}
\begin{example}[Geodesics on a Sphere]
    Consider the two-dimensional unit sphere $S^2$ in $\mathbb R^3$.
    Use the spherical polar coordinates
    $$x=\sin\theta\sin\phi,y=\sin\theta\cos\phi,z=\cos\theta,\theta\in [0,2\pi),\theta\in[0,\pi)$$
    The line element on $S^2$ inherited from $\mathbb R^3$ then gives
    $$\mathrm ds^2=\mathrm dx^2+\mathrm dy^2+\mathrm dz^2=\mathrm d\theta^2+\sin^2\theta\,\mathrm d\phi^2$$
    A path restricted on the sphere can then be parameterized in terms of $\theta,\phi$.
    Suppose we parameterize the curve by $\phi(\theta)$, then the length functional is
    $$F[\phi]=\int_{\theta_1}^{\theta_2}\sqrt{1+(\phi^\prime)^2\sin^2\theta}\,\mathrm d\theta$$
    We observe that the integral inside does not depend on $\phi$, so we can rewrite the equation to
    $$\frac{\phi^\prime\sin^2\theta}{\sqrt{1+(\phi^\prime)^2\sin^2\theta}}=\frac{\partial f}{\partial \phi^\prime}=\text{const.}=K$$
    Seperating the variables yields
    $$(\phi^\prime)^2=\frac{K^2}{\sin^2\theta(\sin^2\theta-K^2)}\implies\phi=\pm\int\frac{K\,\mathrm d\theta}{\sin\theta\sqrt{\sin^2\theta-K^2}}$$
    which shall produce two solutions, each going one way round.
    To evaluate the integral, we do the substitution $u=\cot\theta$ which produces
    $$\pm\frac{\sqrt{1-K^2}}{K}\cos(\phi-\phi_0)=\cot\theta$$
    for a constant $\phi_0$.
    By considering the geometrical meaning of this equation, it then follows that $\phi(\theta)$ describes a great circle, i.e. a circle in $\mathbb R^3$ which exists as the intersection of $S^2$ and a plane that goes through the origin.
\end{example}
Now, consider for general case of $f(x,y,y^\prime)$, we have
\begin{align*}
    \frac{\mathrm d}{\mathrm dx}\left( f-y^\prime\frac{\partial f}{\partial y^\prime} \right)&=\frac{\partial f}{\partial x}+\frac{\partial f}{\partial y}y^\prime+\frac{\partial f}{\partial y^\prime}y^{\prime\prime}-y^\prime\frac{\mathrm d}{\mathrm dx}\frac{\partial f}{\partial y^\prime}-y^{\prime\prime}\frac{\partial f}{\partial y^\prime}\\
    &=y^\prime\left( \frac{\partial f}{\partial y}-\frac{\mathrm d}{\mathrm dx}\frac{\partial f}{\partial y^\prime} \right)+\frac{\partial f}{\partial x}
    &=\frac{\partial f}{\partial x}
\end{align*}
If $y$ satisfies the Euler-Lagrange Equation.
So if $f$ does not depend explicitly on $x$, then the above indicates that
$$f-y^\prime\frac{\partial f}{\partial y^\prime}=\text{const.}$$
\begin{example}[The Brachistochrone Problem]
    Consider the functional in the Brachistochrone Problem we defined before, with the initial point assumed to be the origin:
    $$F[y]=\frac{1}{\sqrt{2g}}\int_0^\beta\frac{\sqrt{1+(y^\prime)^2}}{\sqrt{-y}}\,\mathrm dx$$
    As $f$ in this case does not depend explicitly on $x$, we know that
    $$\frac{\sqrt{1+(y^\prime)^2}}{\sqrt{-y}}-y^\prime\frac{y^\prime}{\sqrt{1+(y^\prime)^2}\sqrt{-y}}=\text{const.}=K$$
    So
    $$y^\prime=\pm\frac{\sqrt{1+K^2y}}{K\sqrt{-y}}\implies x=\pm K\int\frac{\sqrt{-y}}{\sqrt{1+K^2y}}\,\mathrm dy$$
    Set $y=-K^{-2}\sin^2(\theta/2)$, so $\mathrm dy=-K^{-2}\sin(\theta/2)\cos(\theta/2)$, so
    \begin{align*}
        x&=\pm K\int(-1)\frac{1}{K^3}\frac{\sin^2(\theta/2)\cos(\theta/2)}{\sqrt{1-\sin^2(\theta/2)}}\,\mathrm d\theta\\
        &=\mp\frac{1}{2K^2}\int(1-\cos\theta)\,\mathrm d\theta\\
        &=\mp\frac{1}{2K^2}(\theta-\sin\theta)+C
    \end{align*}
    where $C$ is a constant.
    By our initial condition, $\theta(0)=0$, so $C=0$.
    Substitute $\theta$ for both $x,y$ and we get the parameterised equations
    $$\begin{cases}
        x=(\theta-\sin\theta)/(2K^2)\\
        y=-K^{-2}\sin^2(\theta/2)
    \end{cases}$$
    which is the equation of a cycloid, i.e. the path traversed by a fixed point on a wheel which rolls on the $x$-axis.
    Hence, the Brachistochrone is a cycloid.
\end{example}
\subsection{Fermat's Principle}
Fermat's Principle postulates that light (or sound) travels along paths between two points that are stationary points of the time variation.
\footnote{In its original form, however, it said that light travels the path that requires the least time, which is not necessarily true in all cases.}
Suppose the light way is described by $y=y(x)$, then the time functional is
$$F[y]=\int\frac{\mathrm dl}{c}=\int_\alpha^\beta\frac{\sqrt{1+(y^\prime)^2}}{c(x,y)}\,\mathrm dx$$
where $c$ is the speed of light in the medium.\\
First assume that $c=c(x)$ does not depend on $y$, then $f$ does not explicitly depend on $y$, in which case we have the first integral
$$\frac{y^\prime}{\sqrt{1+(y^\prime)^2}c(x)}=\frac{\partial f}{\partial y^\prime}=\text{const.}$$
Suppose the light ray has an initial incident angle of $\theta_1$ upwards, then $\tan\theta_1=y^\prime(\alpha)$.
Let $\tan\theta=y^\prime$ in general, then the above first integral indicates that $\sin\theta/c(x)$ is constant.
This is Snell's Law.
If $c$ is increasing, then the path is concave and if $c$ is decreasing it is convex.
Also, if the light goes through a barrier, to the left of which $c$ is constant at $c_F$ and to the right of which $c_S$ with $c_S<c_F$, then the light ray will refract in the way we all expect.