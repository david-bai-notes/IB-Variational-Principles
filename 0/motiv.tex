\section{Motivation}
The history of variational principles can be dated back to a pretty concrete classical problem known as the Brachistochrone problem:
\begin{example}[The Brachistochrone Problem]
    Consider a particle moving under the influence of gravity on a wire joining points $A$ and $B$.
    What shape of the curve gives the shortest travel time of the particle given that it starts from rest?\\
    Mathematically, we want to minimise the quantity
    $$\tau=\int_A^B\mathrm dt=\int_A^B\frac{\mathrm dL}{v(x,y)}$$
    This problem was first created by Johann Bernoulli in 1696, who posted the problem in a journal as a challenge to the world's mathematcians at the time.
    And the answer given by Newton starts the study of calculus of variations.
    We are not going to fully solve the problem in this section, but instead just give a taste of the whole picture.\\
    Assuming $A$ is the origin and $B=(x_2,y_2)$
    Using energy conservation,
    $$\frac{1}{2}mv^2+mgy=0\implies v=\sqrt{-2gy}$$
    Take $y$ as a function of $x$, then we are aiming at finding the minima of the functional
    $$\tau[y]=\frac{1}{\sqrt{2g}}\int_0^{x_2}\frac{\sqrt{1+(y^\prime)^2}}{\sqrt{-y}}\,\mathrm dx$$
    subject to
    $$\begin{cases}
        y(0)=0\\
        y(x_2)=y_2
    \end{cases}$$
\end{example}
Another very famous and very useful example is geodesics.
\begin{example}[Geodesics]
    On a surface $\Sigma$, a geodesics $\gamma$ is the path of least length joining two given points.
    If $\Sigma$ is the Euclidean plane, it is well-known that $\gamma$ has to be a straight line.
    Let $D[y]$ denote the length of the curve given by $y$ and assume that the curve traverse as a function $y(x)$ from $x_1,x_2$, then minimising the following integral 
    $$D[y]=\int_{x_1}^{x_2}\sqrt{1+(y^\prime)^2}\,\mathrm dx$$
    subject to $y(x_1)=y_1,y(x_2)=y_2$ would be our aim.
\end{example}
In general, the calculus of variations focus on solving optimisation problems of the a functional (i.e. functions from a space of functions to the reals)
$$F[y]=\int_{x_1}^{x_2}f(x,y(x),y^\prime(x))\,\mathrm dx$$
among all (sufficiently smooth) functions $y$ subject to initial conditions.
\begin{example}[Examples of Functionals]
    1. For a function $y(x)$, we can define the functional calculating the area under the curve by setting $f(x,y,y^\prime)=y$.\\
    2. We can also define the functional calculating length by
    $$f(x,y,y^\prime)=\sqrt{1+(y^\prime)^2}$$
\end{example}
Notationally, we write $C(\mathbb R)$ as the space of continuous functions from $\mathbb R$ to $\mathbb R$.
We write $C^k(\mathbb R)$ to denote the space of $k$-time differentiable functions with continuous $k^{th}$ derivative.
And $C^k_{\alpha,\beta}(\mathbb R)$ is the space of functions $[\alpha,\beta]\to\mathbb R$ that are $C^k$ on $[\alpha,\beta]$ and vanishes at $\alpha,\beta$.
One should know that these are these are all (infinite-dimensional) vector spaces over $\mathbb R$, the detail of analysis of which will be covered in functional analysis contexts.\\
The reason why the course is called ``variational principles'' instead of ``calculus of variations'' is because what we are studying are principles in nature and laws of physics that follow from extremising functionals.
\begin{example}[Fermat's Principle]
    Light that travels between two points follow the path that extremise the travel time.
\end{example}
\begin{example}[Principle of Least Action]
    Let $T,V$ be the kinetic and potential energies.
    We define the functional on a path
    $$S[\gamma]=\int_{t_1}^{t_2}(T-V)\,\mathrm dt$$
    then the path that a particle travels from $t=t_1$ to $t=t_2$ is the one that extremises $S$.
    Once we have developed our theory further, one can show that Newton's Second Law follows from this principle.\\
    Leibniz's take on this principle is that we live in ``the best of all possible worlds''.
    In some sense, we go from science to something like the field of theology.
    Of course, this is not in the scope of this course.\\
    Richard Feynman's take on this principle is that it is ``wrong''.
    Indeed, in quantum physics, nothing takes a definitive path, but every paths are possible with some probability.
    The principle under this framework, instead, is that the stationary path of a particle is, in fact, the interference along all paths.
\end{example}
What we shall show in this course are the follows:
First, we will see some necessary conditions for a function to be an extrema, like the Euler-Lagrange equations.
Secondly, we will bring this principle on problems in geometry, physics, and problem with constraints (like the isoperimetric inequality).
We will also talk about what is called the ``second variation'' as an analog of the second derivative.