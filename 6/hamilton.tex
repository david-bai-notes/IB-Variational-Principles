\section{Hamilton's Equations}
Recall that the Lagrangian is defined by $L=L(\underline{q},\underline{\dot{q},t})=T-V$ where $\underline{q}$ is either the path of one particle or the path of many particles considered together in the configuration space.
\begin{definition}
    The Hamiltonian is the Legendre transform of $L$ wrt the velocity $\underline{\dot{q}}=\underline{v}$, so
    $$H(\underline{q},\underline{p},t)=\sup_{\underline{v}}(\underline{p}\cdot\underline{v}-\underline{L})$$
\end{definition}
The component $\underline{p}$ here is understood to be a generalised momentum.
So if $L$ is nice enough, we can write $H=\underline{p}\cdot\underline{v}-L(\underline{x},\underline{v},t)$ where $\underline{v}(\underline{p})$ is the solution to
$$p_i=\frac{\partial L}{\partial \dot{q_i}}$$
\begin{example}
    Consider $T=m|\underline{\dot{q}}|^2/2, V=V(\underline{q})$, so $\underline{v}=\underline{p}/m$, therefore
    $$H(\underline{q},\underline{p},t)=\frac{1}{2m}|\underline{p}|^2+V(\underline{q})$$
    So the Hamiltonian arises as the total energy.
\end{example}
What happened to the Euler-Lagrange equations?
We have $H=p_i\dot{q}^i-L(q^i,\dot{q}^i,t)$ where the summation in the first term is implied.
Suppose $L$ satisfies the Euler-Lagrange equations, we have
\begin{align*}
    \mathrm dH&=\frac{\partial H}{\partial q_i}\,\mathrm dq_i+\frac{\partial H}{\partial p_i}\,\mathrm dp_i+\frac{\partial H}{\partial t}\,\mathrm dt\\
    &=p_i\,\mathrm d\dot{q}^i+\dot{q}^i\,\mathrm dp_i-\frac{\partial L}{\partial q^i}\,\mathrm dq^i-\frac{\partial L}{\partial \dot{q}^i}\,\mathrm d\dot{q}^i-\frac{\partial L}{\partial t}\,\mathrm dt\\
    &=\dot{q}^i\,\mathrm dp_i-\dot{p}_i\,\mathrm dq^i-\frac{\partial L}{\partial t}\,\mathrm dt
\end{align*}
Since $\partial L/\partial \dot{q}^i=p_i$
So by comparing the differentials,
$$\dot{q}^i=\frac{\partial H}{\partial p_i},\dot{p}_i=-\frac{\partial H}{\partial q^i},\frac{\partial H}{\partial t}=-\frac{\partial L}{\partial t}$$
These equations are called the Hamilton's Equations.\\
Assume there is no explicit time dependence, then the system consists of $2n$ first-order ODEs, so we need to specify $q^i(0),p_i(0)$ for $i=1,\ldots,n$ for a typical system.
The solution curves to the Hamilton's equations is a trajectory in $2n$ dimensional ``phase space''.\\
Alternatively, Hamilton's equations also arise from the stationary points of a functional
$$S[\underline{q},\underline{p}]=\int_{t_1}^{t_2}f(\underline{q},\underline{p},\underline{\dot{q}},\underline{\dot{p}},t)\,\mathrm dt,f=\dot{q}^ip_i-H(\underline{q},\underline{p},t)$$
which variation wrt $p_i$ is then
$$\frac{\partial f}{\partial p_i}-\frac{\mathrm d}{\mathrm dt}\frac{\partial f}{\partial \dot{p}_i}=0\implies \dot{q}^i=\frac{\partial H}{\partial p_i}$$
and variation wrt $q_i$ is
$$\frac{\partial f}{\partial q^i}-\frac{\mathrm d}{\mathrm dt}\frac{\partial f}{\partial \dot{q}^i}=0\implies \dot{p}_i=-\frac{\partial H}{\partial q_i}$$
Which are just the Hamilton's equations.\\
Hamilton's equatios (or the Hamiltonian formulation of dynamics) actually allows one to extend the theory of classical dynamics to quantum settings, as observed by Paul Dirac in 1926.
So the Hamiltonian formalism can be seen as a bridge between classical and quantum physics.